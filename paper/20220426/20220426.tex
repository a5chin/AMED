\documentclass[a4j]{ujarticle}
\renewcommand{\baselinestretch}{0.85}
\usepackage[top=1.5cm, bottom=1.5cm, left=1.5cm, right=1.5cm]{geometry}
\usepackage{xcolor}
\usepackage[dvipdfmx]{graphicx, hyperref}
\usepackage{listings}
\usepackage{multirow}
\usepackage{siunitx}
\usepackage{subfig}
\usepackage{url}
\usepackage{listings}
\usepackage{caption,stackengine}


\colorlet{punct}{red!60!black}
\definecolor{background}{HTML}{EEEEEE}
\definecolor{delim}{RGB}{20,105,176}
\colorlet{numb}{magenta!60!black}

\newcommand{\Sref}[1]{\mbox{\ref{sec:#1}}}
\newcommand{\Tref}[1]{\mbox{表\ref{tab:#1}}}
\newcommand{\Eref}[1]{\mbox{式(\ref{eq:#1})}}
\newcommand{\Fref}[1]{\mbox{図\ref{fig:#1}}}
\renewcommand{\lstlistingname}{ソースコード}
\newcommand{\Lref}[1]{\mbox{ソースコード\ref{lst:#1}}}
\newcommand{\bhline}[1]{\noalign{\hrule height #1}}

\lstdefinelanguage{json}
{
    basicstyle=\normalfont\ttfamily,
    numbers=left,
    numberstyle=\scriptsize,
    stepnumber=1,
    numbersep=8pt,
    showstringspaces=false,
    breaklines=true,
    backgroundcolor=\color{background},
    literate=
     *{:}{{{\color{punct}{:}}}}{1}
      {,}{{{\color{punct}{,}}}}{1}
      {\{}{{{\color{delim}{\{}}}}{1}
      {\}}{{{\color{delim}{\}}}}}{1}
      {[}{{{\color{delim}{[}}}}{1}
      {]}{{{\color{delim}{]}}}}{1},
}
\lstset{
	frame=tRBl,
	captionpos=b,
	numbers=left,
	tabsize=4,
    columns=[l]{fullflexible},
    breaklines=true,
}

\hypersetup{
	setpagesize=false,
	bookmarksnumbered=true,
	bookmarksopen=true,
	colorlinks=true,
	linkcolor=black,
	citecolor=black
}

\begin{document}
    \begin{flushright}
        MDLab GM資料\\
        22年4月26日(火)
    \end{flushright}

    \begin{center}
        {\Large	腹部超音波画像からの腫瘍検出}
    \end{center}

    \begin{flushright}
        {\large B3  原 英吾}\\
    \end{flushright}

    \section{研究背景および目的}
    \begin{figure}[h]
        \begin{minipage}{.59\textwidth}
            \begin{itemize}
                \item 背景
                \begin{itemize}
                    \item 検査実施者は器具の操作と診断を同時に行わなければならず高難易度
                    \item 肝臓は沈黙の臓器と呼ばれ初期には自覚症状がほとんどない
                    \begin{itemize}
                        \item 症状を自覚している時には重症化している場合が多い
                    \end{itemize}
                    \item 機械学習による診断のサポート
                    \begin{itemize}
                        \item 提供されているデータには,\Fref{ex}の様に明らかなラベル不足のある画像が存在する
                    \end{itemize}
                \end{itemize}
                \item 目的
                \begin{itemize}
                    \item 既存の研究を踏まえたモデルの精度向上
                    \begin{itemize}
                        \item noisy label\footnotemark[1]による精度低下の改善
                    \end{itemize}
                    \item 超音波支援システムの開発
                    \begin{itemize}
                        \item 早期発見につながると良い
                    \end{itemize}
                \end{itemize}
            \end{itemize}
        \end{minipage}
        \begin{minipage}{.39\textwidth}
            \centering
            \includegraphics[width=.9\linewidth]{../fig/pseudo_a.png}
            \caption{ラベル不足のある診断画像例}
            \label{fig:ex}
        \end{minipage}
    \end{figure}

\footnotetext[1]{今回は\Fref{ex}の様なアノテーションが不足しているものを指す}
\addtocounter{footnote}{1}

    \section{これまでの研究のまとめ}
        \begin{itemize}
            \item データセット
            \begin{itemize}
                \item 国立研究開発法人日本医療研究開発機構(AMED)\footnote{\url{https://www.amed.go.jp/}}が提供している延べ8万枚に及ぶ以下のデータが付随
                \begin{itemize}
                    \item 腹部超音波画像,ROI
                    \item 年齢,性別
                \end{itemize}
                \begin{figure}[h]
                    \centering
                    \subfloat[性別毎の画像枚数]{\includegraphics[width=.24\linewidth]{../fig/sex_a.pdf} \label{fig:sex}}
                    \subfloat[診断名毎の年齢分布]{\includegraphics[width=.24\linewidth]{../fig/age_a.pdf} \label{fig:age}}
                    \subfloat[診断名毎の画像サイズの分布]{\includegraphics[width=.24\linewidth]{../fig/area_a.pdf} \label{fig:area}}
                    \subfloat[診断名毎のbboxの割合]{\includegraphics[width=.24\linewidth]{../fig/ratio_bbox_a.pdf} \label{fig:ratio}}
                    \caption{データセットにおけるデータの分布}
                \end{figure}
                \item 性別(\Fref{sex})
                \begin{itemize}
                    \item hcc(肝細胞癌)は男性が罹患しやすい
                    \begin{itemize}
                        \item 昔は男性の方が飲酒・タバコが多く癌に罹りやすかったという時代背景があるかもしれない
                    \end{itemize}
                    \item hemangioma(血管腫)は女性が罹患しやすい
                    \item meta(転移性肝癌)は他の症状よりも少ない
                \end{itemize}
                \item 年齢(\Fref{age})
                \begin{itemize}
                    \item cyst(単純嚢胞),hemangioma(血管腫)の分布にははあまり特徴がない
                    \item hemangioma(血管腫)は比較的若年層でも罹患する
                    \item meta(転移性肝癌)における0歳はラベルミスである可能性が高い
                    \item hcc(肝細胞癌)は比較的高齢者が罹患しやすい
                \end{itemize}
                \item 画像サイズ(\Fref{area})
                \begin{itemize}
                    \item hemangioma(血管腫)は比較的画像サイズが統一されている
                    \begin{itemize}
                        \item 腫瘍の大きさが血管に依存するためあまり偏りが生じていない?
                    \end{itemize}
                \end{itemize}
                \item bboxの画像に占める割合(\Fref{ratio})
                \begin{itemize}
                    \item cyst(単純嚢胞)は他の診断と比べてbboxの割合が低い($\frac{1}{2}$程度)である
                    \item hcc(肝細胞癌)は画像に占めるbboxの割合が高い
                \end{itemize}
            \end{itemize}
        \end{itemize}

        \begin{itemize}
            \item 症状毎の特徴を調査
            \begin{figure}[h]
                \centering
                \subfloat[cyst(単純嚢胞)]{\includegraphics[width=.24\linewidth]{../fig/cyst.png} \label{fig:cyst}}
                \subfloat[hemangioma(血管腫)]{\includegraphics[width=.24\linewidth]{../fig/hemangioma.png} \label{fig:hemangioma}}
                \subfloat[meta(転移性肝癌)]{\includegraphics[width=.24\linewidth]{../fig/meta.png} \label{fig:meta}}
                \subfloat[hcc(肝細胞癌)]{\includegraphics[width=.24\linewidth]{../fig/hcc.png} \label{fig:hcc}}
                \caption{症状毎における腫瘍の超音波画像}
            \end{figure}
            \begin{itemize}
                \item cyst(単純嚢胞) (\Fref{cyst})
                \begin{itemize}
                    \item 液体が貯留されている状態
                    \item 症状がでないことが多いため大きな腫瘍になって発見されることが多い
                    \item 嚢胞の内腔に向けて増殖するため転移することは少ない
                \end{itemize}
                \item hemangioma(血管腫) (\Fref{hemangioma})
                \begin{itemize}
                    \item 肝臓にできる良性腫瘍の中で最も多い
                    \item 女性ホルモンが原因で女性が罹患しやすいと言われているが詳しくは解明されていない
                    \item 血管が無数に絡み合うことによって出来た血管の塊であることから血流が遅いという特徴がある
                    \item 他の臓器に浸潤したり転移することは無いと言われている
                \end{itemize}
                \item meta(転移性肝癌) (\Fref{meta})
                \begin{itemize}
                    \item 門脈を介して大腸癌などの消化器癌から転移する割合が多い
                    \item 類似したエコーパターンをもつ腫瘤が多発してみられることが多い
                \end{itemize}
                \item hcc(肝細胞癌) (\Fref{hcc})
                \begin{itemize}
                    \item 肝臓にできる悪性腫瘍の中で最も多いと言われている
                    \item 約90%がウイルス感染症が原因
                    \begin{itemize}
                        \item B型肝炎ウイルス(HBV)が約20\%
                        \item C型肝炎ウイルス(HCV)が約70\%
                    \end{itemize}
                \end{itemize}
            \end{itemize}
        \end{itemize}

        \begin{itemize}
            \item データクレンジング
            \begin{enumerate}
                \item $400 \times 400$ 以下の画像の除外
                \item Perceptual Hashを利用した類似画像の除外
                \item 青色や黄色のスケールの除去
            \end{enumerate}

            \item 提供されているデータをCOCODatasetの形式に変換
            \begin{itemize}
                \item train data : test data : val data = 67122 : 8390 : 8391
                \item 見やすいようにインデントしたものも作成\footnote{\url{//aka/work/hara.e/AMED/lib/dataset/annotations/train_large.json}など}
            \end{itemize}

            \item 1クラス (腫瘍) での学習
            \begin{itemize}
                \item 実験のパラメーター
            \end{itemize}

            \begin{minipage}{.95\textwidth}
                \stackunder{
                    \begin{minipage}[h]{.44\textwidth}
                        \setcounter{table}{0}
                        \centering
                        \label{tab:conditions}
                        \begin{tabular}{c|c}
                            seed & 0 \\
                            model & YOLOX-s \\
                            pretrained & ImageNet \\ \hline
                            data数 & 67122 \\
                            batch\_size & 64 \\
                            total\_epoch & 50 \\
                            input\_size & (512, 512) \\ \hline
                            weight\_decay & 0.0005 \\
                            momentum & 0.9 \\
                            scheduler & yoloxwarmcos \\
                            criterion & BCELoss \\ \hline
                            device & gpgpu8 \\
                            計算時間 & 30 min/epoch \\
                        \end{tabular}
                    \end{minipage}
                }
                {
                    \begin{minipage}[h]{.44\textwidth}
                        \captionof{table}{学習に用いた条件}
                    \end{minipage}
                }
                \hfill
                \stackunder{
                    \begin{minipage}[h]{.54\textwidth}
                        \centering
                        \includegraphics[width=\linewidth]{../fig/yoloxs_lr.pdf}
                    \end{minipage}
                }
                {
                    \begin{minipage}[h]{.54\textwidth}
                        \setcounter{figure}{3}
                        \captionof{figure}{yoloxwarmcos}
                    \end{minipage}
                }
            \end{minipage}

            \begin{itemize}
                \item Double Descent\cite{double_descent}が起きている
                \begin{itemize}
                    \item Noisy Labelが最大の要因
                \end{itemize}
            \end{itemize}
            \begin{figure}[h]
                \centering
                \subfloat[AP]{\includegraphics[width=.49\linewidth]{../fig/yoloxs1_AP.pdf} \label{fig:ap}}
                \subfloat[loss]{\includegraphics[width=.49\linewidth]{../fig/yoloxs1_loss.pdf} \label{fig:loss}}
                \caption{YOLOXで1クラスの検出を行った時のAPとloss}
            \end{figure}
            \begin{table}[h]
                \centering
                \caption{学習で得られた精度}
                \label{tab:exp}
                \begin{tabular}{ccc|ccc|ccc}
                    & & & & IoU & & & area \footnotemark & \\
                    model & backbone & num\_classes & mAP & AP$_{50}$ & AP$_{75}$ & AP$_S$ & AP$_M$ & AP$_L$ \\ \hline
                    YOLOX\cite{yolox} & Decoupled head & 1 & 0.519 & 0.839 & 0.558 & - & 0.639 & 0.631 \\
                \end{tabular}
            \end{table}

            \item 実験からわかったこと
            \begin{itemize}
                \item \Fref{loss}から
                \begin{itemize}
                    \item Noisy Labelの影響が大きい
                    \begin{itemize}
                        \item IoUの精度が悪い
                        \item Double Descentが起きている
                    \end{itemize}
                    \item クラスの誤分類は少ない
                    \begin{itemize}
                        \item 1クラス分類なので当たり前
                    \end{itemize}
                \end{itemize}
            \end{itemize}
        \end{itemize}
        \footnotetext{Small $< 32^2 <$ Medium $<96^2<$ Large}

    \section{前回のGMからの進捗}
        \begin{itemize}
            \item プログラムを作成しIoU毎にPR曲線を描画
            \begin{figure}[h]
                \centering
                \includegraphics[width=.4\linewidth]{../fig/pr-curve.pdf}
                \caption{1クラス検出におけるPR曲線}
                \label{fig:pr}
            \end{figure}
            \begin{itemize}
                \item \Fref{pr}からRecallが低いことがわかる
                \begin{itemize}
                    \item 医療用でRecallが低いのは問題
                    \item Recallが低いということは腫瘍を検出できていないということ
                \end{itemize}
            \end{itemize}

            \item 目加田先生と研究の今後の方針について話し合った
            \begin{itemize}
                \item ゴールとしては悪性腫瘍 (見逃し厳禁) と良性腫瘍を見分けること
                \item 検出器 (YOLOX\cite{yolox}) と分類器 (自己教師あり学習) を合わせたモデルを構築して精度を測る
                \begin{itemize}
                    \item 分類器について今のところSimCLR\cite{simclr} (対照学習) を検討中
                \end{itemize}
            \end{itemize}
        \end{itemize}

    \section{今後の課題\&スケジュール}
        \begin{itemize}
            \item 5/17までに
            \begin{enumerate}
                \item 4クラスの検出を300epochで学習させる
                \item 悪性腫瘍・良性腫瘍のPrecision,Recallを算出する
                \item 対照学習のコードを書き始める
            \end{enumerate}
            \item 5/31までに
            \begin{itemize}
                \item 対照学習の結果を (遅くとも6/14までに) 出すことを目標にする
            \end{itemize}
        \end{itemize}

    \begin{thebibliography}{9}
        \bibitem{double_descent} Anonymous authors. \href{https://openreview.net/attachment?id=B1g5sA4twr&name=original_pdf}{DEEP DOUBLE DESCENT: WHERE BIGGER MODELS AND MORE DATA HURT} ICLR 2020, 2020.
        \bibitem{yolox} Zheng Ge, Songtao Liu, Feng Wang, Zeming Li, and Jian Sun. \href{https://arxiv.org/pdf/2107.08430.pdf}{YOLOX: Exceeding YOLO Series in 2021}, 2021.
        \bibitem{simclr} Ting Chen, Simon Kornblith, Mohammad Norouzi, Geoffrey Hinton. \href{https://arxiv.org/pdf/2002.05709.pdf}{A Simple Framework for Contrastive Learning of Visual Representations}, 2020.
    \end{thebibliography}
\end{document}